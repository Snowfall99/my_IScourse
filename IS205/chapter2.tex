\documentclass[a4paper, 12pt]{article}
\usepackage{ctex}
\usepackage{amsmath}
\title{Chapter 2}
\author{Xuan}
\begin{document}
    \maketitle
    \section{离散信源熵与互信息}
    \subsection{信源熵}
    $H(X)=\sum_ip(x_i)\log_2\frac{1}{p(x_i)}=-\sum_ip(x_i)\log_2p(x_i)$
    \paragraph{性质:}
    \begin{enumerate}
        \item $H(X)\ge0$
        \item $H(X)\le\log_2|X|$
    \end{enumerate}
    当某一符号$x_i$的概率为$p_i$为零时,$p_i\log_2p_i$在熵公式中无意义,为此规定这时的$p_i\log_2p_i$为零。
    当信源$X$中只含一个符号$x$时,必定有$p(x)=1$,此时信源熵$H(x)$为零,是确定信源。
    \subsubsection{证明$H(x)\le\log_2|X|$:}
    \paragraph{方法一:} 求偏导,略
    \paragraph{方法二:} 利用相对熵\\
    假设q服从均匀分布,即$q_i=\frac{1}{|X|}$
    \begin{align}
        D(p||q)&=\sum_ip_i\log_2\frac{p_i}{q_i}\\
        &=\sum_ip_i\log_2p_i-\sum_ip_i\log_2q_i\\
        &=-H(x)-\sum_ip_i\log_2q_i\ge0\\
        &=-H(x)+\sum_ip_i\log_2|X|\ge0
    \end{align}
    所以我们可以得到
    \[H(x)\le\log_2|X|\]
    \subsection{相对熵}
    \subsubsection{证明$D(p||q)\ge0$}
    \paragraph{下凸函数的性质} $pf(x_1)+(i-p)f(x_2)\ge f(px_1+(1-p)x_2)$
    \paragraph{Jensen's Inequality:} 对于下凸函数而言,
    \[Ef(X)\ge f(EX)\]
    上凸函数反之。
    \begin{enumerate}
        \item $|X|=2$时,由下凸函数性质可证
        \item 数学归纳法
    \end{enumerate}
    Assume $|X|=k-1$, Jensen's Inequality holds\\
    Prove $|X|=k$, Inequality holds as well.\\
    Assume $\sum_{i=1}^{k-1}p(x_i)=q=1-p(x_k)$
    \begin{align}
        Ef(X)&=p(x-k)f(x_k)+\sum_{i=1}^{k-1}p(x_i)f(x_i)\\
        &=p(x_k)f(x_k)+q\sum_{i=1}^{k-1}\frac{p(x_i)}{q}f(x_i)\\
        &=p(x_k)f(x_k)+(1-p(x_k))f(\sum_{i=1}^{k-1}\frac{p(x_i)}{1-p(x_k)}x_i)\\
        &\ge f(p(x_k)x_k+\sum_{i=1}^{k-1}p(x_i)x_i)\\
        &=f(EX)
    \end{align}
\end{document}