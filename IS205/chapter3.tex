\documentclass[a4paper, 12pt]{article}
\usepackage{ctex}
\usepackage{amsmath}
\title{Chapter 3: 信道与信道容量}
\author{Xuan}
\begin{document}
    \maketitle
    \section{信道的基本概念}
    \subsection{二进制离散信道(BSC)}
    该信道模型的输入和输出信号的符号数都是2,即$X\in A={0,1}$和$Y\in B={0, 1}$,转移概率为
    \begin{equation}
        \begin{aligned}
            p(Y=0|X=1)&=p(Y=1|X=0)=p\\
            p(Y=1|X=1)&=p(Y=0|X=0)=1-p
        \end{aligned}
    \end{equation}
    \subsection{加性高斯白噪声信道(AWGN)}
    \[Y=X+G\]
    G是一个零均值、方差为$\sigma^2$的高斯随机变量,当$X=a_i$给定后,Y是一个均值为$a_i$、方差为$\sigma^2$的高斯随机变量
    \[p_Y(y|a_i)=\frac{1}{\sqrt{2\pi\sigma^2}}e^{-(y-a_i)^2/2\sigma^2}\]
\end{document}