\documentclass[a4paper, 12pt]{article}
\usepackage{ctex}
\usepackage{amsmath}
\title{Exercises}
\author{Xuan}
\begin{document}
    \maketitle
    \section{散度 Divergence}
    \paragraph{Problem 1}用另外一种方式证明$D(p||q)\ge 0$
    \begin{equation}
        \begin{aligned}
            -D(p||q)&=\sum p(x)ln\frac{q(x)}{p(x)}\\
            &\le \sum p(x)(\frac{q(x)}{p(x)}-1)\\
            &=\sum (q(x)-p(x))\\
            &=1-1\\
            &=0
        \end{aligned}
    \end{equation}
    hint:$lnx\le x-1$
    \paragraph{Problem 2}混淆增加熵率

    证明:$\vec(p)=(p_1,p_2,...,p_m)$,$\vec{q}=(p_1,p_2,...,\frac{p_i+p_j}{2},...,\frac{p_j+p_i}{2},...,p_m)$,
    有$\H(\vec{q})\ge H(\vec{p})$
    \begin{equation}
        \begin{aligned}
            H(\vec{q})-H(\vec{p})&=-2*\frac{p_i+p_j}{2}log\frac{p_i+p_j}{2}+p_ilogp_i+p_jlogp_j\\
            &=-p(log\frac{p_i+p_j}{2}-\frac{p_i}{p}logp_i-\frac{p_j}{p}logp_j)(p=p_i+p_j)\\
            &=-p(log\frac{p_i+p_j}{2}-logp-(\frac{p_i}{p}log\frac{p_i}{p})-(\frac{p_j}{p}log\frac{p_j}{p}))\\
            &=-p(log\frac{p_i+p_j}{2}-logp+p_1log\frac{1}{p_1}+p_2log\frac{1}{p_2})\\
            &\ge -p(log\frac{p}{2}-logp+log(p_1*\frac{1}{p_1}+p_2*\frac{1}{p_2}))\\
            &=0
        \end{aligned}
    \end{equation}
    hint:Jensen's Inequality:$\lambda f(x_1)+(1-\lambda)f(x_2)\le f(\lambda x_1+(1-\lambda)x_2)$, for concave function
    \section{互信息}
    \paragraph{Problem 3} 设X,Y,Z,T,满足$H(T|X)=H(T),H(T|X,Y)=0,H(T|Y)=H(T),H(Y|Z)=0,H(T|Z)=0$,
    
    证明:(1)$H(T|X,Y,Z)=I(Z,T|X,Y)=0$
    \[H(T|X,Y,Z)=H(T|X,Y)-I(T;Z|X,Y)\ge 0\]
    Because $H(T|X,Y)=0$, we can come to the fact that $I(T;Z|X;Y)=0$, as a consequence we can conclude that $H(T|X,Y,Z)=0$

    (2)$I(X;T|Y,Z)=I(Y;T|X,Z)=0$
    \[I(X;T|Y,Z)=H(T|Y,Z)-H(T|X,Y,Z)\]
    From (1), we have the fact that $H(T|X,Y,Z)=0$. Moreover, we have the fact that 
    $H(T|Z)=0=I(T;Y|Z)+H(T|Y,Z)$ and $I(T;Y|Z) \ge 0$, such that $H(T|Y,Z)=0$. As a consequence,
    we can conclude that $I(X;T|Y,Z)=0$. We can draw thE conclusion that $I(Y;T|X,Z)=0$ with 
    the same method.

    (3)$I(X;Y|Z,T)=0$
    \[I(X;Y|Z,T)=H(Y|Z,T)-H(Y,X|Z,T)\]
    Because $H(Y|Z)=0$, we can easily have $H(Y|Z,T)=0$. Meanwhile, $I(X;Y|Z,T)\ge 0$,
    such that $H(Y|X,Z,T)$ can only be 0. As a consequence, we can conclude that $I(X;Y|Z,T)=0$

    (4)$I(X;Z)\ge H(T)$
    \section{信道容量}
    \paragraph{Problem 4} Binary Erasure Channel(二进制擦拭信道),问:找到$p(x)$,使得$C=Imax(X;Y)$, 图见exe1\_4

    \begin{equation}
        \begin{aligned}
            I(X;Y)&=H(X)-H(X|Y)=H(Y)-H(Y|X)\\
            p(y)&=\sum_x p(x,y)=\sum_x p(y|x)p(x)\\
            H(Y)&=-\sum_y p(y)logp(y)\\
            &=-\sum_y(\sum_xp(y|x)p(x))log(\sum_xp(y|x)p(x))
        \end{aligned}
    \end{equation}
    \begin{equation}
        \begin{aligned}
            p(Y=0)&=p_0(1-p-\epsilon)+(1-p_0)p=p_0(1-29-\epsilon)+p\\
            p(Y=1)&=p_0p+(1-p_0)(i-p-\epsilon)=(1-p-\epsilon)+(2p+\epsilon-1)p_0\\
            p(Y=\epsilon)=p_0\epsilon+(1-p_0)\epsilon=\epsilon
        \end{aligned}
    \end{equation}
    From equation(4) above, we can get that $p(Y=0)+p(Y=1)=1-\epsilon$. Apparently,
    H(Y) reaches its maximum when $p_0=\frac{1}{2}$.
    \[
        H(Y|X)=\sum p(x)H(Y|X=x)=p(0)H(Y|X=0)+p(1)H(Y|X=1)    
    \]
    From the transition matrix, we can get that $H(Y|X=0)=H(Y|X=1)$. So that, we 
    have $H(Y|X)=H(Y|X=0)=H(Y|X=1)$, which is only related to $\epsilon$ and p,
    so $I(X;Y)$ reaches its maximum, when $H(Y)$ reaches its maximum, which has nothing
    to do with $H(Y|X)$.

    \paragraph{Problem 5} (转移概率矩阵和及串联信道)问X$\rightarrow$Y信道的转移概率,图见exe1\_5

    method 1:
    \begin{equation}
        \begin{aligned}
            p(x=0,y=0)&=p(x=0)p(y=0|y=0)\\
            &=p(x=0)(p(y=0,z=0|x=0)+p(y=0,z=1|x=0))\\
            &=p(x=0)(p(y=0|z=0)p(z=0|x=0)+p(y=0|z=1)p(z=1|x=0))
        \end{aligned}
    \end{equation}
    So we can get,
    \begin{equation}
        \begin{aligned}
            p(y=0|x=0)&=(1-p_2)(1-p_1)+p_1p_2\\
            p(y=1|x=0)&=(1-p_1)p_2+(1-p_2)p_1\\
            p(y=0|x=1)&=p_1(1-p_2)+(1-p_1)p_2\\
            p(y=1|x=1)&=(1-p_1)(1-p_2)+p_1p_2
        \end{aligned}
    \end{equation}
    method 2: 见图片
\end{document}